\chapter{Obtaining and Installing CalChart}\label{software}

\section{Obtaining the Software}\label{obtaining}

The latest version of CalChart may be obtained from the official
CalChart web page:

\begin{verbatim}
http://sourceforge.net/projects/calchart
\end{verbatim}

There are currently versions for Windows and Mac.
UNIX and Linu platforms are also possible, but you'll
have to compile it yourself.

\section{Installing the Software}\label{installing}

\subsection{Windows Installation}\label{wininstall}

The Windows installer can be downloaded from:

\begin{verbatim}
http://sourceforge.net/projects/calchart
\end{verbatim}

Once downloaded, open the installer and follow the instructions.

\subsection{MacOSX Installation}\label{macosxinstall}

To install on MacOSX, download the dmg file and open.  Drag the CalChart app
to the Applications folder.

\section{Printer Configuration}\label{winprinter}

In general, there are two ways of printing, Native Printing, and Post-Script
Printing.  For windows, Native Printing may have sub-optimal results, so you
may want to do Post-Script printing.  With Post-Script printing CalChart can
print to PostScript printers (such as laser printers).  You can also  print
to a PostScript file and use ghostscript (a free PostScript interpreter) to
print to another printer.  On a mac, the post-script file can be viewed with Preview.

