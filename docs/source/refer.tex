\chapter{Reference}\label{refer}

\section{Printed Continuity Codes}\label{pcontcodes}

\begin{tabular}{|l|p{7cm}|l|}\hline
\row{{\bf CODE}&{\bf MEANING}}\hline\hline
\row{$\%\%$NAME&Begin a sheet titled NAME}
\row{$\backslash$po&plainman}
\row{$\backslash$pb&backslashman}
\row{$\backslash$ps&slashman}
\row{$\backslash$px&xman}
\row{$\backslash$so&solidplainman}
\row{$\backslash$sb&solidbackslashman}
\row{$\backslash$ss&solidslashman}
\row{$\backslash$sx&solidxman}
\row{$\backslash$bs&start boldface}
\row{$\backslash$be&end boldface}
\row{$\backslash$is&start italics}
\row{$\backslash$ie&end italics}
\row{$<$&line appears only on continuity sheets}
\row{$>$&line appears only on stuntsheets}
\ruledrow{$\sim$&center line}
\end{tabular}

The symbols $<>\sim$ must appear in that order at the start of the line.
The file must begin with $\%\%$.

Use tabs to align the measure number, symbol, and continuity columns.
