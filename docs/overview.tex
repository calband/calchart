\chapter{Overview}\label{overview}

CalChart Version 3.2 is a update of of CalChart Version 3 (the version
that has been used by the Cal Band from 1991 to the 2008).  The major
improvements have been to port it to the MacOS X platform, integrating closer
with the native windowing system functionality like file history and printing.
It retains backward compatility with CalChart Version 3 files, though older
CalChart version are not supported.

CalChart, like all software, may contain issues and bugs.  And like all 
software will greatly benefit from feedback from
users.  Please document any bugs that you encounter on the CalChart developer 
website at:

\begin{verbatim}
http://sourceforge.net/projects/calchart
\end{verbatim}

Also, we would very much like to hear any suggestions you may have to make
CalChart even more useful.

This chapter gives some hints on getting to know Version 3.2.  Much is
very similar to Version 3, and for users already familiar with Version 3 may 
not even notice a difference.

\section{Basics}\label{basics}

The interface is mouse-driven and windows-based.  Many features cause a new
windows to appear in addition to the main window displaying the stuntsheet.
The user may have as many windows open as desired at any one time.  It
is possible to have many windows of the same kind open simultaneously,
and it is possible to have different windows displaying different shows
simultaneously.

To scroll the stuntsheet viewer, just move the mouse around the field
display area of the window while holding the control key; scrolling happens
automatically.

Various operations can be performed by either clicking on an icon just
above the stuntsheet display area or by using a pull-down menu.  Some
features have keystroke commands also.

Left-drag the cursor in the stuntsheet editor to select a set of points.
Once points are selected many features can be applied to them.
Use Shift-left-drag to add points to an already selected block.
Use Alt-left-drag to toggle a points selection.

Use the Undo feature under the Edit pull-down menu to undo actions.
Repeating this causes older and older actions to be undone.  The user
can retrace his steps all the way back to the most recent save
(providing that the undo buffer, which defaults to 100 items, is big enough).

Use the icons just above the stuntsheet display area to navigate through
the show.  Use the slider to adjust the magnification of the display.

Window sizes can be adjusted by dragging the corner of the window
around.

\section{Creating a Show}\label{newshow}

Starting the program causes a small window to appear.  From the File pull-down
menu you may open an existing show via the "Open..." item (or the Alt-O hotkey),
or you may create a new show via the "New" item (or the Alt-N hotkey).  You may 
also choose from the list of files to quickly load a previous open filed.

Creating a new show will start the "New Show Setup Wizard", which is intended
to make creating your new show easier.  The wizard can be used to determine the
number of points, the kind of labels, and the kind of background.  You must
press finish to have the changes take effect, but you choose cancel and 
then modify any of these settings later on.

\section{Moving Points}\label{movement}

There are several move methods of moving points around the field.  You select
different move methods by clicking their toolbar icon.  After a move has been
completed, the move method will default back to Translate Points.

\subsection{Translate Points}\label{translatepoints}

This movement command is used to move individuals or blocks to a new location
on the field.

To individual move, left-drag on a point, then move the cursor
to the destination, then release the left button.

To block move, first select
the block, then hold shift and move any one point of the block as if it
were an individual move.

\subsection{Move Points Into a Line}\label{movepointsintoaline}

This movement command is used to move a block into a straight line.

First select the block of points you wish to have placed in a line.  Then
left-click the point where the first point in the line should start.  While
holding down the left-button, drag the mouse to where the second point in the
line should be.  Release the mouse button and CalChart will move all of the 
points into a straight line from that point.

\subsection{Rotate Block}\label{rotateblock}

This movement command is used to move rotate a block around a point.

First select the block of points you wish to have rotated.  Left-click the 
pivot point where the block should be rotated around.  On the second left-click
a guiding circle should appear.  While holding the left-button, drag around
the circle to choose the desired angle.  Release the mouse button and CalChart
will rotate the block around the pivot point by the desired angle.

\subsection{Shear Block}\label{shearblock}

This movement command is used to shear a block across around a point.

First select the block of points you wish to have sheared.  Left-click the 
pivot point where the block should be sheared around.  On the second left-click
a guiding line should appear.  The line must extented at a 90 degree angle 
from the start point to the pivot point.   While holding the left-button, drag
the line to the desired shearing amount.  Release the mouse button and CalChart
will shear the block around the pivot point by the desired amount.

\subsection{Reflect Block}\label{reflectblock}

This movement command is used to reflect a block around a line.

First select the block of points you wish to have reflected.  Left-click and
drag a reflection line.  Release the mouse button and CalChart
will reflect all the points their mirror location on the other side of the line.

\subsection{Resize Block}\label{resizeblock}

This movement command is used to resize a block, to make it larger or smaller.

First select the block of points you wish to have resized.  Left-click the 
central point where the block should be resized around.  Left-click and
drag a resizing line.  Release the mouse button and CalChart
will resize the block around the central point by the desired amount.

\subsection{Genius Move}\label{geniusmove}

This movement command is used to translate a block of points by describing the
general shape you wish the block to attend.

First select the block of points you wish to have moved.  Left-click and drag
3 lines that generally describe the shape you wish the block to take.
Release the mouse button and CalChart will check to see if it is a valid move,
and if so, will apply a genious move to the block.

\section{Labels and Symbols}\label{symbols}

Select a set of points, then click on one of the icons just above the
stuntsheet display area for the desired action.

\section{Editing Continuity}\label{editcont}

Open an Edit Continuity window (under the Edit pull-down menu) to type
continuity commands.  Set continuity groups by selecting points and then
pressing the appropriate button.  New continuity groups can also be
created from the menu.

\section{Printed Continuity}\label{printcont}

This is not yet finished but it mostly works for viewing the continuity
as it will be printed.  Open a Printed Continuity window under the Edit
pull-down menu.

\section{Animation}\label{animwin}

Open an Animation window (under the Window pull-down menu) to view animation.
Use the icons to navigate through the show and the slider to set the tempo.
Click on the check-box labeled Collisions to enable detection of collisions.
If you make changes to the show, you can select Regenerate from the menu
to recalculate everything.  When errors are encountered while generating
the animation, a window will appear with the types of errors.  You can
click on an error to select the points that caused this error.

There are many significant improvements over Version 2.  First of all,
there are no more animation files.  Everything is handled dynamically.
You can also select points and follow them in the animation window
(e.g. you can follow the percussion line.)

\section{Reference Points}\label{refpoints}

Click on the Ref pop-up menu to select a different reference group.
