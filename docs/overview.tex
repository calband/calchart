\chapter{Overview}\label{overview}

CalChart Version 3 is a complete rewrite of CalChart Version 2 (the version
that has been used by the Cal Band from 1991 to the present).  The major
improvements are a completely new mouse-driven windows-based interface and
a number of significant internal changes that are only relevant to
programmers and system administrators.  The internal changes include a
drastic improvement in the way the program saves data, the use of a 
more modern programming language, and the ability to work with Novell
networks and Windows NT.

Version 3 has reached a point in its development where it is ready for
large-scale testing.  There are still many bugs, some of which are already
known but most of which are not, and there are still some features that
are not completed.

The finalization of this version will greatly benefit from feedback from
users.  Please document any bugs that you encounter and send them to
gurk@calband.berkeley.edu.  Also, we would very much like to hear any
suggestions you may have to make CalChart even more useful.

This chapter gives some hints on getting to know Version 3.  Much is
self-explanatory for a user already familiar with Version 2, but much is
new.

\section{Basics}\label{basics}

The interface is mouse-driven and windows-based.  Many features cause a new
windows to appear in addition to the main window displaying the stuntsheet.
The user may have as many windows open as desired at any one time.  It
is possible to have many windows of the same kind open simultaneously,
and it is possible to have different windows displaying different shows
simultaneously.

To scroll the stuntsheet viewer, just move the mouse around the field
display area of the window; scrolling happens automatically.

Various operations can be performed by either clicking on an icon just
above the stuntsheet display area or by using a pull-down menu.  Some
features have keystroke commands also.

Left-drag the cursor in the stuntsheet editor to select a set of points.
Unlike Version 2, nothing will happen immediately except that the
selected points will change color.  Once points are selected many
features can be applied to them.  Use Shift-left-drag to add points to
an already selected block.

Use the Undo feature under the Edit pull-down menu to undo actions.
Repeating this causes older and older actions to be undone.  The user
can retrace his steps all the way back to the most recent save
(providing that the undo buffer, which defaults to 50K, is big enough).

Use the icons just above the stuntsheet display area to navigate through
the show.  Use the slider to adjust the magnification of the display.

Window sizes can be adjusted by dragging the corner of the window
around.

\section{Creating a Show}\label{newshow}

Starting the program causes a small window to appear.  You may either
load an existing show or create a new one from this window.  You may
also use the File pull-down menu from an open show to load, save, clear,
etc. in the obvious way.  After creating a new show, you'll need to pull
up an Information window (under the Window pull-down menu).

The Information window can be used to determine the number of points, the
kind of labels, and the kind of background.  In order for changes to be
enacted, you must click on the button at the top.  Note this window can be
used even after the show is first created.

\section{Moving Points}\label{movement}

Smart move and genius move are not yet implemented, so the only tools for
moving points are individual move, block move, and smart move.  To
individual move, left-drag on a point, then move the cursor to the
destination, then release the left button.  To block move, first select
the block, then hold shift and move any one point of the block as if it
were an individual move.  To smart move, first select the appropriate
selection order from the options menu.  {\it Nearest} can be used to
selecting winding chains of points.  Select the points, beginning the
drag near the first point.  Any additional selections will be appended to
the end of the selection list.  Click on the smart move icon in the
toolbar and drag a line from the first location to the second.  The
other points will be placed with equal spacing.

\section{Labels and Symbols}\label{symbols}

Select a set of points, then click on one of the icons just above the
stuntsheet display area for the desired action.

\section{Editing Continuity}\label{editcont}

Open an Edit Continuity window (under the Edit pull-down menu) to type
continuity commands.  Set continuity groups by selecting points and then
pressing the appropriate button.  New continuity groups can also be
created from the menu.

\section{Printed Continuity}\label{printcont}

This is not yet finished but it mostly works for viewing the continuity
as it will be printed.  Open a Printed Continuity window under the Edit
pull-down menu.

\section{Animation}\label{animwin}

Open an Animation window (under the Window pull-down menu) to view animation.
Use the icons to navigate through the show and the slider to set the tempo.
Click on the check-box labeled Collisions to enable detection of collisions.
If you make changes to the show, you can select Regenerate from the menu
to recalculate everything.  When errors are encountered while generating
the animation, a window will appear with the types of errors.  You can
click on an error to select the points that caused this error.

There are many significant improvements over Version 2.  First of all,
there are no more animation files.  Everything is handled dynamically.
You can also select points and follow them in the animation window
(e.g. you can follow the percussion line.)

\section{Reference Points}\label{refpoints}

Click on the Ref pop-up menu to select a different reference group.
