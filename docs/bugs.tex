\chapter{Guide to Testing CalChart}\label{testing}

Most of the testing I have done is with small, specific examples.  I'd
like people to put it through normal use.  Start by charting a short
show from scratch.

There are also two components that I think should be heavily tested: the
undo/redo system and the animation system.

The default is to buffer 50K of undo information.  I've not tested the
program significantly when the buffer is full.  Therefore, I'd like you
to use the program for a long time continuously and then undo as much as you
can.  Then try to redo everything and if everything is the same.  You
should also try to use these options as much as possible when charting.

The animation system is by far the most complicated part of the
program.  I've tried to make the commands work the same as in the
original program, but there are many variations that still need to be
tested.  I'd also like to know if collision detecting slows down the
animation on slower machines (like the band's computers).

You can begin by downloading all of the old shows from the CalChart web
page and go make sure they are all animated correctly.  Some of them may
also give errors under the old version of CalChart, so try them with it
as well.  You should also save each show and reload it and check for any
changes.  You should make sure that any collisions detected in the old
version show up in this one.  I may divide up the work of testing the
old shows.

Most importantly, you should test every feature of the continuity language
(even commands like EITHER and features like variables).  Try making a
show that uses obscure commands or does things in weird ways (like use
complicated functions to compute the direction).  Try animating this
show with the old version of CalChart (you'll have to create it with the
old version or re-enter it) and look for any inconsistencies.  You
should refer to the section on \helpref{animation commands}{animcont}
to make sure the documentation matches the behavior of the program.

And finally, don't only test correct continuity, but also look at how
the new and old versions handle errors.  Using zeros or incorrect values
or the wrong number of beats is a good place to start.

\section{About Reporting Bugs}\label{reportingbugs}

Mail bug reports to gurk@calband.berkeley.edu.  Please try to include
helpful information.  "It crashed" is not going to help me fix bugs.
Please try to reproduce the bug, and give me instructions on how to
duplicate it, or include a show that illustrates the problem.  Include
the date from the About window so I know what version you're using.  Comments
or suggestions are also appreciated and should be mailed to
calchart-discussion@calband.berkeley.edu.

\section{Known Problems}\label{problems}

\begin{itemize}\itemsep=0pt
\item Only prints to PostScript printers
\item Edits in continuity are not immediately registered
\item Documentation is sorely lacking
\item Animation tempo cannot be adjusted while playing
\item Redrawing is slow
\item Handling of points off the field (and in endzones) could be better
\end{itemize}

\section{Unimplemented Features}\label{notdone}

\begin{itemize}\itemsep=0pt
\item Printed continuity is not saved
\item Printed continuity is not editable
\item Genius movement is not implemented
\item Unused continuities are not automatically deleted
\item Moving the pointer over a reference point doesn't highlight the
corresponding point
\item Yard lines can only be moved by editing the configuration file
\end{itemize}
