\chapter{Animation Commands}\label{animcont}
\section{DIRECTIONS}\label{directions}

CalChart uses the mathematical convention to assign degree values to
directions.  Any real number is acceptable for a direction, including
negative values.  The program uses the following symbols to represent the
most commonly used directions:

\begin{tabular}{|l|p{5cm}|l|}\hline
\row{{\bf NAME}&{\bf DIRECTION}&{\bf VALUE}}\hline\hline
\row{N&north&0}
\row{NW&northwest&45}
\row{W&west&90}
\row{SW&southwest&135}
\row{S&south&180}
\row{SE&southeast&225}
\row{E&east&270}
\ruledrow{NE&northeast&315}
\end{tabular}

A special symbol for the "direction of flow" is called DOF.  DOF contains
the direction of the last movement or marktime (note that it is undefined
if none has occurred).

Any function which returns a direction may be used in place of a direction;
see the section on \helpref{FUNCTIONS}{functions}.


\section{FUNCTIONS}\label{functions}

A function is used in place of a numerical parameter and takes on a
specific value depending on its parameters (if it needs any); this is
called "returning a value."  A typical use of a function is:
\begin{verbatim}
  MT STEP(2 2 R1) S
\end{verbatim}
Note that a function must always be part of another command.

Some functions take parameters, and if there is more than one, they are
separated by a space.  Unlike procedures, the parameter list is surrounded
by parentheses.  In the example above, the STEP function takes three
parameters: 2 (a number), 2 (another number) and R1 (a point).  It returns
a number of steps (see the \helpref{STEP}{step} function for more details).

Note that there are two types of parameters: numerical and point.
When one is expected, the other is not allowed.

These are the functions in CalChart:

\subsection{DIR}\label{dir}

Format: DIR(\verb$<point>$)

DIR returns the direction that P would have to travel to reach \verb$<point>$.
The value is expressed in degrees.  Note also that this direction follows
the math convention (see the section on \helpref{DIRECTIONS}{directions}).

For example,
\begin{verbatim}
  DIR(NP)
\end{verbatim}
will return the direction to travel from P to NP.

DIR is precisely equivalent to:
\begin{verbatim}
  DIRFROM(P <point>)
\end{verbatim}

\subsection{DIRFROM}\label{dirfrom}

Format: DIRFROM(\verb$<start point>$ \verb$<end point>$)

DIRFROM returns the direction needed to get from \verb$<start point>$ to
\verb$<end point>$. The value is expressed in degrees and using the math
convention (see \helpref{DIRECTIONS}{directions} for more details).

For example,
\begin{verbatim}
  DIRFROM(R1 NP)
\end{verbatim}
will return the direction one would have to go to reach NP from R1.

A related function, \helpref{DIR}{dir}, assumes P for the \verb$<start point>$.

\subsection{DIST}\label{dist}

Format: DIST(\verb$<point>$)

DIST returns the distance from P (the current position) to \verb$<point>$.
If the direction from P to \verb$<point>$ is NW, SW, SE or NE, then the
distance is given in diagonal-military steps; otherwise it is given in
high-steps.

For example,
\begin{verbatim}
  DIST(NP)
\end{verbatim}
returns the distance from P to NP.

A related function, \helpref{DISTFROM}{distfrom}, always returns high-steps.

\subsection{DISTFROM}\label{distfrom}

Format: DISTFROM(\verb$<point 1>$ \verb$<point 2>$)

DISTFROM returns the distance, in units of one high-step, from \verb$<point 1>$
to \verb$<point 2>$.

For example,
\begin{verbatim}
  DISTFROM(R1 NP)
\end{verbatim}
will return the distance between R1 and NP.

A related function, \helpref{DIST}{dist}, assumes P for \verb$<point 1>$ and
uses diagonal-military steps if a diagonal direction is used.

\subsection{EITHER}\label{either}

Format: EITHER(\verb$<direction 1>$ \verb$<direction 2>$ \verb$<reference>$)

EITHER returns either \verb$<direction 1>$ or \verb$<direction 2>$, whichever
is closer to the direction from P to \verb$<reference>$.  When the two
directions are equally far from the direction from P to \verb$<reference>$,
then \verb$<direction 1>$ is returned.

For example,
\begin{verbatim}
  EITHER(N S R1)
\end{verbatim}
executed when R1 is north of P will return N; otherwise it will return S.

\subsection{OPP}\label{opp}

Format: OPP(\verb$<direction>$)

OPP simply returns a direction opposite to the one specified.

Example:
\begin{verbatim}
  OPP(DIR(NP))
\end{verbatim}
returns the direction which is directly away from NP.

See \helpref{DIRECTIONS}{directions} for details on the mathematics of this.

\subsection{STEP}\label{step}

Format: STEP(\verb$<beats>$ \verb$<blocksize>$ \verb$<start point>$)

STEP is used in conjunction with MT to implement step-drills,
where one person starts, and then some steps (say x) later the next leaves,
and x steps after that the next leaves, etc.  STEP returns the number of
counts for the given person to marktime in relation to the given start
point; however, it can be used for other things besides step-drills.

The \verb$<beats>$ parameter specifies how many beats are to elapse between
people leaving, and the \verb$<counts>$ parameter specifies how far apart the
people are from each other.

The \verb$<start point>$ is a point which is considered the beginning of
the drill (e.g. the first to leave).

For example,
\begin{verbatim}
  MT STEP(2 4 R1) W
\end{verbatim}
will make each person in a 4-step block leave 2 counts after his neighbor;
the first person to leave is at the same location as R1.  While they are
waiting to go, the people will marktime west.

\section{NUMBERS}\label{numbers}

When a function or procedure calls for a number (for example a stepsize,
direction, or angle), the best method of expressing this value is with
a constant, such as HS (for a stepsize) or NW (for a direction).
See the sections on DIRECTIONS, BEATS, and STEPSIZES for lists
of constants.

Alternatively, a function can be used (see that section for more
information), or a simple number can be used, such as 12.5 or -180.

Also, a variable can be used, such as A or X.  See the section on
\helpref{VARIABLES}{variables} for details.

Finally, these different types of numerical expressions can be combined
using standard arithmetic symbols:

\begin{tabular}{|l|p{7cm}|}\hline
\row{addition&\verb$<num>$ + \verb$<num>$}
\row{subtraction&\verb$<num>$ - \verb$<num>$}
\row{multiplication&\verb$<num>$ * \verb$<num>$}
\row{division&\verb$<num>$ / \verb$<num>$}
\ruledrow{negation&- \verb$<num>$}
\end{tabular}

These symbols are interpreted in stardard arithmetic order, unless they are
grouped with parentheses.

Numerical expressions cannot be used when a point is expected.

\section{POINTS}\label{points}

To specify a point, one of these symbols MUST be used:

\begin{tabular}{|l|p{9cm}|}\hline
\row{{\bf NAME}&{\bf POINT}}\hline\hline
\row{SP&location on the current stuntsheet}
\row{P&current location}
\row{NP&location on the next stuntsheet}
\row{R1&first reference point on the current stuntsheet}
\row{R2&second reference point}
\ruledrow{R3&third reference point}
\end{tabular}

First, note that P changes with every beat, but the others are constant
throughout a single stuntsheet.

Second, points differ from numerical expressions in that they cannot be
combined by arithmetic operators, and cannot be used when a numeric
parameter is expected.

\section{PROCEDURES}\label{procedures}

Procedures are used to give marching instructions, like moving from one
position to another, rotating, or marking time.

A typical procedure call is:
\begin{verbatim}
  EVEN REM-12 NP
\end{verbatim}
The first word is the name of a procedure.  Following are the parameters
that tell the procedure what to do (if it needs any).  Parameters are of
two types: POINTS and NUMBERS (see those sections for more information).
When one is expected, the other is not allowed.  Parameters should be
separated from each other by a space.  Unlike functions, the parameter list
is not surrounded by parentheses.

The above example has
\begin{verbatim}
  REM-12
\end{verbatim}
as the first (numerical) parameter, and
\begin{verbatim}
  NP
\end{verbatim}
as the second (point) parameter.

These are the procedures that may be used in CalChart:

\subsection{BLAM}\label{blam}

Format: BLAM

BLAM uses all remaining beats, and any stepsize necessary to move the current
point to NP.  Note that all the beats are used, and the movement
may occur off the grid.  BLAM is equivalent to:
\begin{verbatim}
  MARCH DIST(NP)/REM REM DIR(NP) or EVEN REM NP
\end{verbatim}
except when REM is 0, in which case it is equivalent to
MAGIC \verb$<destination>$.

\subsection{COUNTERMARCH}\label{countermarch}

Format: COUNTERMARCH \verb$<point 1>$ \verb$<point 2>$ \verb$<steps>$ \verb$<dir 1>$ \verb$<dir 2>$ \verb$<beats>$

COUNTERMARCH is a complicated procedure which implements a countermarch in
the Cal band style.

The movement commences at \verb$<point 1>$ by moving \verb$<steps>$ in
direction \verb$<dir 1>$, then moving in direction \verb$<dir 2>$, then
moving the opposite of \verb$<dir 1>$ until reaching \verb$<point 2>$,
then continuing \verb$<steps>$ in the direction opposite of \verb$<dir1>$,
then moving the opposite of \verb$<dir 2>$, then in direction \verb$<dir 1>$
until \verb$<point 1>$ is reached again.

This pattern is continued until the specified number of beats have been used.

A standard high-step countermarch of length 16 and R1 and R2 as the two
corners would look like this:
\begin{verbatim}
  COUNTERMARCH R1 R2 1 N E 16
\end{verbatim}

In most cases, \helpref{DMCM}{dmcm} or \helpref{HSCM}{hscm} can be used
more easily.

A diagram follows:
                                         
\begin{verbatim}
       <dir 1>
       --------->

------------------------------------------>*------->        <dir 2>
^                                      <point 1>   |           |
|     <point 2>                                    V           |
<-------*<------------------------------------------           V
\end{verbatim}

\subsection{DMCM}\label{dmcm}

Format: DMCM \verb$<point 1>$ \verb$<point 2>$ \verb$<beats>$

DMCM executes a diagonal-military countermarch in the Cal Band style.

The movement commences at \verb$<point 1>$ by moving one step either NW, NE,
SE or SW, then moving two steps E or W, then moving opposite to the first
direction until reaching \verb$<point 2>$, then continuing one step in the
same direction, then moving the two steps E or W, then in the initial
direction until \verb$<point 1>$ is reached again.

This pattern is continued until the specified number of beats have been used.

A diagonal-military countermarch of length 16 and R1 and R2 as the two
corners would look like this:
\begin{verbatim}
  DMCM R1 R2 16
\end{verbatim}

This is a specialized case of \helpref{COUNTERMARCH}{countermarch}, as is
\helpref{HSCM}{hscm}.

\subsection{DMHS}\label{dmhs}

Format: DMHS \verb$<destination>$

DMHS marches a two-part course to \verb$<destination>$.  First, the
direction will be NW, NE, SW, or SE, and stepsize will be DM.  Then,
the direction will be N, S, E, or W, and stepsize will be HS.  The
specific directions are selected so that the path traveled is as short
as possible.

This procedure is similar to \helpref{FOUNTAIN}{fountain}, but DMHS does
not require explicit directions.  It is also similar to \helpref{HSDM}{hsdm}.

\subsection{EVEN}\label{even}

Format: EVEN \verb$<steps>$ \verb$<destination>$

EVEN marches towards the point \verb$<destination>$ and determines the
stepsize so that it arrives at the point \verb$<destination>$ in exactly
the number of steps specified.  It is precisely the same as:
\begin{verbatim}
  MARCH DIST(<destination>)/<steps> <steps> DIR(<destination>)
\end{verbatim}

\subsection{EWNS}\label{ewns}

Format: EWNS \verb$<destination>$

EWNS marches a two-part course, starting with an east-west trajectory, and
ending with a north-south trajectory.  EWNS is equivalent to:
\begin{verbatim}
  FOUNTAIN E N <destination>
\end{verbatim}

A similar procedure is \helpref{NSEW}{nsew}.

\subsection{FOUNTAIN}\label{fountain}

Format: FOUNTAIN \verb$<direction 1>$ \verb$<direction 2>$ \verb$<destination>$

or FOUNTAIN \verb$<direction 1>$ \verb$<direction 2>$ \verb$<stepsize 1>$ \verb$<stepsize 2>$ \verb$<destination>$

FOUNTAIN marches a two-part course, starting in \verb$<direction 1>$ and
ending in \verb$<direction 2>$, and arriving at \verb$<destination>$.
If necessary, the opposite of \verb$<direction 1>$ will be used instead
of \verb$<direction 1>$, and the opposite of \verb$<direction 2>$ will
be used instead of \verb$<direction 2>$.

If the stepsizes are not given, a value of 1 will be used.

Note that if either trajectory is unnecessary, it will be skipped.

In most cases, \helpref{NSEW}{nsew}, \helpref{EWNS}{ewns},
\helpref{DMHS}{dmhs}, or \helpref{HSDM}{hsdm} can be used more easily.

\subsection{FM}\label{fm}

Format: FM \verb$<steps>$ \verb$<direction>$

FM is similar to \helpref{MARCH}{march} in that it moves P in the
specified direction for the specified number of steps.  However, the
stepsize is inferred from direction (see \helpref{STEPSIZES}{stepsizes}
for details).

An error will occur if the number of beats remaining is less than
\verb$<steps>$.

\subsection{FMTO}\label{fmto}

Format: FMTO \verb$<destination>$

FMTO marches in a straight line to the point \verb$<destination>$.  It
infers the stepsize and then determines how many steps to take using
that stepsize.  (See the section on \helpref{STEPSIZES}{stepsizes} for
more information on guessing).

\subsection{GRID}\label{grid}

Format: GRID \verb$<grid scale>$

GRID is a special command that moves P (see \helpref{MAGIC}{magic})
without taking any beats.  It moves P onto the specified scale of grid,
where \verb$<grid scale>$ is interpreted in high-steps.

For example,
\begin{verbatim}
  GRID 2
\end{verbatim}
will move P so that it is on a two-step high-step grid.

This command is useful for correcting small errors in special continuities.

\subsection{HSCM}\label{hscm}

Format: HSCM \verb$<point 1>$ \verb$<point 2>$ \verb$<beats>$

HSCM executes a high-step countermarch in the Cal Band style.

The movement commences at \verb$<point 1>$ by moving one step either N or S,
then moving two steps E or W, then moving opposite to the first direction
until reaching \verb$<point 2>$, then continuing one step in the same
direction, then moving the two steps E or W, then in the initial direction
until \verb$<point 1>$ is reached again.

This pattern is continued until the specified number of beats have been used.

The upper-right and lower-left corners of the countermarch block should be
specified with reference points.

A high-step countermarch of length 16 and R1 and R2 as the two corners would
look like this:
\begin{verbatim}
  HSCM R1 R2 16
\end{verbatim}

This is a specialized case of \helpref{COUNTERMARCH}{countermarch}, as is
\helpref{DMCM}{dmcm}.

\subsection{HSDM}\label{hsdm}

Format: HSDM \verb$<destination>$

HSDM marches a two-part course to \verb$<destination>$.  First, the
direction will be N, E, W, or S, and stepsize will be HS.  Then, the
direction will be NW, NE, SE, or SW, and stepsize will be DM.  The
specific directions are selected so that the path traveled is as short
as possible.

This procedure is similar to \helpref{FOUNTAIN}{fountain}, but HSDM does
not require explicit directions.  It is also similar to \helpref{DMHS}{dmhs}.


\subsection{MAGIC}\label{magic}

Format: MAGIC \verb$<destination>$

MAGIC is special: P is moved directly to \verb$<destination>$, using the
{\it last beat} that was marched.  In other words, it uses up no beats.

MAGIC is generally used when animation either is impossible (e.g. Script
Cal) or is undesired.

\subsection{MARCH}\label{march}

Format: MARCH \verb$<stepsize>$ \verb$<steps>$ \verb$<direction>$

or MARCH \verb$<stepsize>$ \verb$<steps>$ \verb$<direction>$
\verb$<face direction>$

MARCH will move the current point (P) in the specified direction for the
specified number of beats.  Each step will be of size
\verb$<stepsize>$.  The point will face the direction of movement unless
a direction is given.

An error will occur if there are not enough counts remaining.

Note that if \verb$<steps>$ is negative, the person will march backwards!

\subsection{MT}\label{mt}

Format: MT \verb$<beats>$ \verb$<direction>$

Marktime does not move P but simply leaves it in place, facing
\verb$<direction>$, for the specified number of beats.

An error will occur if \verb$<beats>$ exceeds the number of beats remaining.

\subsection{MTRM}\label{mtrm}

Format: MTRM \verb$<direction>$

MTRM is similar to \helpref{MT}{mt}, in that P remains stationary,
facing \verb$<direction>$.  However, the number of beats is assumed to
be the number of beats remaining.  Note that because of this, all the
remaining beats are used up.

MTRM is precisely equivalent to:
\begin{verbatim}
  MT REM <direction>
\end{verbatim}

\subsection{NSEW}\label{nsew}

Format: NSEW \verb$<destination>$

NSEW plots a two part course, starting with a north-south trajectory, and
ending in an east-west trajectory.  NSEW is equivalent to:
\begin{verbatim}
  FOUNTAIN N E <destination>
\end{verbatim}
because FOUNTAIN can reverse directions (see that section for details).

A similar procedure is \helpref{EWNS}{ewns}.

\subsection{ROTATE}\label{rotate}

Format: ROTATE \verb$<angle>$ \verb$<steps>$ \verb$<pivot point>$

ROTATE will rotate P around the \verb$<pivot point>$ for the given
number of steps.  The total angle traversed will be \verb$<angle>$ (given in
degrees).

Note that angle is interpreted in the standard mathematical way:
  Positive angles rotate counter-clockwise
  Negative angles rotate clockwise

Another note:  If \verb$<pivot point>$ coincides with P, then the person will
rotate around himself (i.e. stand in one place and turn slowly).

\section{STEPS}\label{steps}

There are several ways to specify the number of beats to be used.  An
explicit integer may be specified, such as 8.

Any numerical function or variable may also be used (see those sections).

Additionally, the special symbol REM, which stands for remaining, always
contains the number of beats which remain to be used on the current
stuntsheet.  Thus, REM changes with every beat.

\section{STEPSIZES}\label{stepsizes}

Stepsizes are specified in units of one high-step.  The following constants
are supplied step-sizes:

\begin{tabular}{|l|p{5cm}|p{7cm}|}\hline
\row{{\bf NAME}&{\bf STEPSIZE}&{\bf VALUE}}\hline\hline
\row{HS&1 high-step&1}
\row{MM&1 mini-military&1}
\row{GV&1 grapevine-step&1}
\row{SH&1 show-high&.5}
\row{JS&1 jerky-step&.5}
\row{M&1 military-step&1.333333 (8/6)}
\ruledrow{DM&1 diagonal-military&1.414214 (the square root of 2)}
\end{tabular}

In addition, any numerical expression may be used (for example, 1.5, or
DIST/8, etc.)

Some procedures determine a stepsize based on direction.  The method is:
  If direction is NW, SW, SE or NE, stepsize is DM
  Otherwise,                        stepsize is HS
Procedures and functions that utilize this method are:
  \helpref{DIST}{dist}, \helpref{STEP}{step},
\helpref{COUNTERMARCH}{countermarch}, \helpref{EWNS}{ewns},
\helpref{FM}{fm}, \helpref{FMTO}{fmto}, \helpref{FOUNTAIN}{fountain},
\helpref{NSEW}{nsew}

\section{VARIABLES}\label{variables}

In order to aid in complex continuity, values can be stored in variables.
There are seven that may be used:
  A, B, C, D, X, Y, Z

There is no specific meaning to any variable, and each can be used for any
purpose.

To place a value in a variable, use an assignment statement; for example:
\begin{verbatim}
  A = 3
\end{verbatim}
This places the value 3 in the variable A; it will remain there until it is
replaced or the execution of the continuity is completed.

To use a variable, simply use the name of the variable in place of a number.
