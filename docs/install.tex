\chapter{Obtaining and Installing CalChart}\label{software}

\section{Obtaining the Software}\label{obtaining}

The latest version of CalChart may be obtained from the official
CalChart web page:

\begin{verbatim}
http://sourceforge.net/projects/calchart
\end{verbatim}

There are currently versions for Windows (XP, Vista, Windows7) and Mac (Leopard,
SnowLeopard and Lion).  UNIX and Linu platforms are also possible, but you'll
have to compile it yourself.

\section{Installing the Software}\label{installing}

\subsection{Windows Installation}\label{wininstall}

You need an unzip program installed.  Unzip the software with the
command:

\begin{verbatim}
unzip calchart.zip
\end{verbatim}
or
\begin{verbatim}
pkunzip -d calchart.zip
\end{verbatim}

\subsubsection{Easy Installation}\label{easyinstall}

For simple installation, choose Run from the Program Manager's menu.
Run the install.exe program you extracted.  Enter the path you want and
click on Okay.  A group called Cal Band will be created with an icon for
CalChart.

\subsubsection{Printer Configuration}\label{winprinter}

In general, there are two ways of printing, Native Printing, and Post-Script
Printing.  For windows, Native Printing may have sub-optimal results, so you
may want to do Post-Script printing.  With Post-Script printing CalChart can
print to PostScript printers (such as laser printers).  You can also  print
to a PostScript file and use ghostscript (a free PostScript interpreter) to s
print to another printer.

\subsection{MacOSX Installation}\label{macosxinstall}

To install on MacOSX, download the dmg file and open.  Drag the CalChart app
to the Applications folder.

\subsubsection{Printer Configuration}\label{macosxprinter}

In general, there are two ways of printing, Native Printing, and Post-Script
Printing.  With Post-Script printing CalChart can
print to PostScript printers (such as laser printers).  You can also  print
to a PostScript file and use Preview to open the file.

