\chapter{Obtaining and Installing CalChart}\label{software}

\section{Obtaining the Software}\label{obtaining}

The latest version of CalChart may be obtained from the official
CalChart web page:

\begin{verbatim}
http://www.calband.berkeley.edu/calchart/
\end{verbatim}

There are currently versions for Windows (95, NT, and Win32s), SunOS,
and Linux.  A version for Macintosh is on the way.  Other UNIX platforms
are also possible, but you'll have to compile it yourself.

If you are running Windows 3.1, you must also download and install
Win32s (if it isn't installed already).  I'm no longer supporting the
version that runs without Win32s.

\section{Installing the Software}\label{installing}

\subsection{Windows Installation}\label{wininstall}

You need an unzip program installed.  Unzip the software with the
command:

\begin{verbatim}
unzip calchart.zip
\end{verbatim}
or
\begin{verbatim}
pkunzip -d calchart.zip
\end{verbatim}

\subsubsection{Easy Installation}\label{easyinstall}

For simple installation, choose Run from the Program Manager's menu.
Run the install.exe program you extracted.  Enter the path you want and
click on Okay.  A group called Cal Band will be created with an icon for
CalChart.

\subsubsection{Advanced Installation}\label{advinstall}

Advanced users can simply create a new icon.  You should copy
ctl3d32.dll to your windows system directory if it isn't already there.

\subsubsection{Math Coprocessor Support}\label{mathcop}

If you have a machine without a math coprocessor (386 or 486SX) you need
to install the math coprocessor emulator.  In the [386enh] section of
the SYSTEM.INI file, add the line:

\begin{verbatim}
device=C:\WINDOWS\SYSTEM\WEMU387.386
\end{verbatim}

And restart Windows.  Actually, I'm not sure if this emulator is necessary.
If anyone doesn't have a math coprocessor, please let me know if
CalChart works without it.

\subsubsection{Printer Configuration}\label{winprinter}

Edit the file runtime$\backslash$config and change the printer device to the
one your printer is connected to.  CalChart can currently only print to
PostScript printers (such as laser printers).  This is not likely to
change in the near future.  You could print to a PostScript file and use
ghostscript (a free PostScript interpreter) to print to another
printer.

\subsection{UNIX Installation}\label{unixinstall}

You need gnuzip installed on your system.  cd to the directory where you
wish to install CalChart.  Uncompress the software with the command:

\begin{verbatim}
gzip -dc chartbin.tar.gz | tar xvf -
\end{verbatim}

To run CalChart, you should include this directory in your path and set
the environment variable $CALCHART\_RT$ to the path of the runtime
directory.

If you downloaded a version with dynamic libraries, you must also
extract the libraries.  Set the variable $LD\_LIBRARY\_PATH$ to this
directory.  If you don't have Motif on your system (e.g. Linux) you
should get the static binaries.

If you get lots of errors printed when starting CalChart, you probably
need to copy XKeysymDB to /usr/lib/X11 or /usr/openwin/lib.  This
location may be overridden with the variable $XKEYSYMDB$.

\subsubsection{Online Help}\label{unixhelp}

You must install wxhelp if you want to use online help.  Install it using
the procedures above and make sure it's in your path.

\subsubsection{Printer Configuration}\label{unixprinter}

Edit runtime/config to set the programs and options used for printing
and previewing.  The defaults are lpr and ghostview.
